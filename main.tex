\documentclass[letterpaper, 12pt]{article}
\usepackage[utf8]{inputenc}
\usepackage[t1]{fontenc}
\usepackage[spanish]{babel}

\title{la crisis de los fundamentos como detonante de la computacion moderna}
\author{Raul Daza Liñan}
\date{\today}

\begin{document}

\maketitle

Desde el siglo XVII las matemáticas ha evolucionado de una forma abismal, gracias al cálculo los matemáticos pudieron resolver muchos problemas que antes no se habían podido afrontar, inventaron matemáticas cada vez más complejas, en contraparte de los avances logrados, los matemáticos se habían olvidado de lo fundamental, las bases de las matemáticas. A finales del siglo XIX Georg Cantor produjo la teoría de conjuntos que parecía ser lo más cercano a las bases de las matemáticas, pero muchos matemáticos consideraban que las bases de las matemáticas era algo más fundamental que la teoría de conjuntos, así dando inicio a la crisis de los fundamentos, que llevaría a pensar en las bases de las matemáticas, esto conduciría a una crisis filosófica que derivaría en el inicio de la computación moderna.

Todo inició cuando Georg Cantor creó la teoría de conjuntos, Cantor se dio cuenta de que los números podían organizarse por grupos y que los grupos de números mas sencillos derivaban en grupos mas complejos. Se puede iniciar con los números de contar, los números naturales; luego usando la operación de la resta entre estos números, por ejemplo 4-5=-1, podemos conseguir un conjunto más complejo, los enteros; jugando con los enteros con la operación de la división, por ejemplo 5/3 = 1.6666…, podemos conseguir los números raciones; así sucesivamente podemos manipular números con operaciones más complejas derivándonos en conjuntos más complejos. Cantor consiguió muchos avances con su teoría de conjuntos, tal como el descubrimiento de los distintos infinitos. Él dio cuenta de esto al estudiar el cardinal (la cantidad de elementos que posee un conjunto) de los conjuntos de números, al comparar los cardinales de varios grupos notó que había infinitos mayores que otros, a estos los organizó de menor a mayor, llamando Álef 0 al primero, al siguiente Álef 1 y así sucesivamente, a estos infinitos los agrupo en los ‘cardinales transinfinitos’. Aunque la teoría de conjuntos iniciaba desde un conjunto muy básico (los números naturales), había matemáticos que consideraban que las matemáticas iniciaban de algo más fundamental. Gottlob Frege creía que la base de las matemáticas era la lógica, el desarrollo un conjunto de sistemas axiomáticos-lógicos que pretendía explicar la aritmética y los conjuntos, pero este sistema no era perfecto, Bertrand Russell descubrió paradojas en este sistema, por ejemplo la paradoja de Russell: si tenemos un grupo, llamémoslo ‘C’, que contiene a todos los grupos que no se contienen a sí mismo, con C tenemos dos casos, primer caso, C no es un grupo que se contiene a si mismo así que C debe pertenecer al grupo al que pertenecen los grupos que no se contienen a sí mismo y ese grupo es C, es decir, C contiene a C, es algo contradictorio ya que habíamos dicho que C no se contiene a si mismo; segundo caso, C es un grupo que se contiene a sí mismo, pero C es el grupo que contiene grupos que no se contienen a sí mismo, así que C no puede contenerse a mismo, llegamos a una contradicción. En los dos casos llegamos a contradicciones lógicas, esta paradoja y otras llevaron a la creación de corrientes matemáticas que proponían una forma de solucionar estos errores, estas corrientes son: el intuicionismo, el logicismo y el formalismo.

El intuicionismo fue creado por Luitzen Egbertus Jan Brouwer, esta corriente proponía volver a trabajar con la aritmética clásica eliminando los números transinfinitos de Cantor y el infinito actual porque consideraban que la forma de como Cantor usó a las matemáticas era la errónea. esta corriente no fue fructífera porque más delante Hilbert demostraría que el infinito actual no generaba contradicciones. 

Otra corriente fue el logicismo, lo inicio Bertrand Russell, esta corriente decía que los fundamentos de las matemáticas era la lógica. El logicismo definió como grupos predicativos a aquellos que no se contienen a sí mismo, y como grupos no-predicativos a aquellos que se contienen a sí mismo. Para los logistas los grupos y razonamientos no-predicativos debían ser eliminados para arreglar las contradicciones producidas en la teoría de grupos y la lógica de Frege, este cambio solucionó con éxito los problemas encontrados en las teorías anteriores, pero no aseguraba que no surgieran otras. 

La tercera corriente es el formalismo, creado por David Hilbert, dado que intuicionismo fracaso y el logicismo no aseguraba evitar contradicciones futuras, el camino quedó libre para el eminente matemático David Hilbert. Hilbert proponía crear un sistema axiomático que genere la matemática completa y consistente, además de que las demostraciones llevadas a cabo sean finitas, este sistema debería tener reglas concretas y precisas escritas en un lenguaje con palabras y gramática no ambiguas, estos axiomas deben ser tan rigurosos que incluso las teorías derivadas de estos podrían ser corroboradas de forma mecánica. El formalismo falló por las demostraciones de matemático Kurt Gödel, quien postuló y demostró dos teoremas de incompletitud, el primero es que cualquier sistema matemático debe ser inconsistente o incompleto, el segundo es que un sistema matemático consistente no puede probar su consistencia, imposibilitando la creación de un sistema axiomático generador de matemáticas consistentes y completas. Unos pocos años después Alan Turing demostró que no todo problema matemático es resoluble con una serie finita de paso, para esto uso una forma mecánica, una maquina a la que llamo ‘maquina automática’ que hoy se conoce como maquina universal de Turing o simplemente máquina de Turing, esta maquina estaba formada por una cinta con casillas, una cabecilla que lea lo que están en las casillas y un programa que le indique a la cabecilla como moverse y actuar de acuerdo a lo que este escrito en la casilla. La maquina de Turing es la base de la computación moderna, los computadores de hoy en día funcionan con el mismo comportamiento de la máquina de Turing.

La crisis de los fundamentos llevo a muchos avances matemáticos, a mostrar que las matemáticas no eran tan infalibles y perfectas como se pensaban, además derivando en la computación moderna que nos ha permitido una capacidad de cálculo mayor a la que el ser humano nunca había tenido. 

\begin{thebibliography}{0}

\bibitem{Nelo Maestre} Nelo Maestre y Ágata Timón. " Así terminó el sueño de las matemáticas infalibles (y de paso, nació la computación moderna)".En (2018). Url: https://www.bbvaopenmind.com/ciencia/matematicas/asi-termino-el-sueno-de-las-matematicas-infalibles/
\bibitem{PBS} PBS Infinite Series. "Crisis in the Foundation of Mathematics | Infinite Series". En (2017). url: https://www.youtube.com/watch?v=KTUVdXI2vng&t=503s
\bibitem{Miguel} Miguel Ángel Morales Medina. "LA PARADOJA DE RUSSELL". En (2006). Url:
https://www.gaussianos.com/la-paradoja-de-russell/
\bibitem{Gregor} Gregory J. Chaitin. "Ordenadores, paradojas y fundamentos de las matemáticas". En (2003). url: http://ciencias.uis.edu.co/fundamentos/doc/investigacion\_y\_ciencia.pdf
\bibitem{undefined} undefined Behavior. "Math's Existential Crisis (Gödel's Incompleteness Theorems)". En (2016). Url: https://www.youtube.com/watch?v=YrKLy4VN-7k
\bibitem{Derivando} Derivando. "¿Qué es una máquina de Turing?". en (2018). Url: https://www.youtube.com/watch?v=iaXLDz\_UeYY
\bibitem{Alejandro} Alejandro Ortiz Fernández. "crisis en los fundamentos de la matematica". en (1988)


\end{thebibliography}


\end{document}
